\section{Noise caused by the sun}\label{ssec:background_noise}
The sun is the most dominant source of unwanted photons at Europa. This section will investigate how much energy is hitting the surface of Europa. 

To calculate that the sun will be modeled as an ideal black body. The spectral irradiance of the sun can be calculated using \cref{eq:spectral_irradiance}.
\begin{align}\label{eq:spectral_irradiance}
I_\lambda(\lambda,T) = \frac{2hc^2}{\lambda^5}\frac{1}{e^{\frac{hc}{\lambda kT}}-1}
\end{align}
where $I_\lambda(v,t)$ is spectral irradiance with unit $W/m^3$. \\
$h$ is the Planck constant\\
$c$ is the speed of light in vacuum \\
$k$ is the Boltzmann Constant \\
$\lambda$ is the wavelength of the electromagnetic radiation\\
$T$ is the absolute temperature of the body\\



The spectral irradiance of the sun is calculated in \cref{tab:sun_irradiation}.

\begin{table}[H]
\centering
\caption{Calculation of sun irradiation}
\label{tab:sun_irradiation}
\begin{tabular}{|l|l|}\hline
    \textbf{Sun irradiation} & \\
    \hline 
    $h$ & $663.00\cdot10^{-36}\,Js$ \\
    $c$ & $300.00\cdot10^6\,m/s$ \\
    $k$ & $13.80\cdot10^{-24}\,j/K$ \\
    $\lambda$ & $850.00\,n m$ \\
    $I_\lambda$ & $15.11\cdot10^{12}\,W/m^3$ \\
    \hline 
\end{tabular}
\end{table}


The next step is to calculate the power emitted by the sun in the specified bandwidth, at the location of Europa. The specified bandwidth is in this case the assumed bandpass filter used on the lens. The emitted power is calculated by modeling the sun as a point source, and then spreading that power over a sphere with a radius equal to the distance between the sun and Europa, as is done in \cref{eq:point_source}.

\begin{align}\label{eq:point_source}
    P_{sun} = I_{\text{sun}} B_\lambda S \frac{r_{\text{sun}}^2}{r_{\text{Europa}}^2}
\end{align}
where $I_{\text{sun}}$ is the spectral irradiance of the sun at the center frequency of the filter, $B_\lambda$ is the bandwidth of the filter in meters, $S$ the surface area of the target area on Europa, $r_{\text{sun}}$ the of radius of the sun, and $r_{\text{Europa}}$ the distance between Europa and the sun. The effective radiance of the background noise at Europa is calculated in \cref{tab:background_power} using \cref{eq:point_source}.

\begin{table}[H]
\centering
\caption{Calculation of background power on target area on Europa}
\label{tab:background_power}
\begin{tabular}{|l|r|}\hline
    \textbf{Background power} & \\
    \hline 
    $I_\lambda$ & $15.11\cdot10^{12}\,W/M^3$ \\
    $B_\lambda$ & $10.00\,n m$ \\
    Surface area & $15625.00\, m^2$ \\
    $r_{sun}$ & $695.70\,\,km$ \\
    $r_{europa}$ & $800.00\cdot10^3\,km$ \\
    $P_B$ & $1.79\,k W$ \\
    \hline 
\end{tabular}
\end{table}


The next step is to calculate the percentage of energy that hits the device when hovering over Europa. The focal length and aperture of the lens will be configured in such a way that the target surface on Europa fills the entire view at the maximum altitude of the Hazard Detection Mode, so that altitude will be chosen to calculate the received noise power. The amount of power received at the lens of the device can be calculated using \cref{eq:power_lens}.

\begin{align}\label{eq:power_lens}
P'_B = \frac{P_B\cdot R_{Europa}\cdot D_l\cdot \text{opacity}}{2r^2}
\end{align}
where $P'_B$ is the power hitting the lens, \\
$P_B$ the noise power on the target area of Europa,\\
$R_{Europa}$ the reflectivity of Europa,\\
$D_l$ the diameter of the lens,\\
opacity the opacity of the lens,\\
and $r$ the altitude of the device. The calculations are performed in \cref{tab:effective_noise_power}.

\begin{table}[H]
\centering
\caption{Pulse frequency for both modes of operation}
\label{tab:effective_noise_power}
\begin{tabular}{|l|r|}\hline
    \textbf{effective noise power} & \\
    \hline 
    $P_B$ & $1.79\,k W$ \\
    $r$ & $500.00\, m$ \\
    $R_{europa}$ & $35.00\, \%$ \\
    Diameter lens $(D_l)$ & $50.00\,m m$ \\
    opacity filter $(L_f)$ & $50.00\, \%$ \\
    opacity optics $(L_l)$ & $14.60\, \%$ \\
    $P_B2$ & $4.56\,\mu W$ \\
    \hline 
\end{tabular}
\end{table}


Finally the power needs to be converted to number of photons. To calculate how many photons bounce from the surface of Europa and actually hit the light. To calculate the amount of photons one needs to know the amount of energy per photon. This can be calculated using \cref{eq:e_photon}. The calculation is performed in \cref{tab:energy_of_photon}.

\begin{align}\label{eq:e_photon}
E_{photon} = \frac{hc}{\lambda}
\end{align}

\begin{table}[H]
\centering
\caption{Pulse frequency for both modes of operation}
\label{tab:energy_of_photon}
\begin{tabular}{|l|r|}\hline
    \textbf{energy of photon} & \\
    \hline 
    $h$ & $663.00\cdot10^{-36}\,Js$ \\
    $c$ & $300.00\cdot10^6\,m/s$ \\
    $\lambda$ & $850.00\,n m$ \\
    $E_{photon}$ & $234.00\cdot10^{-21}\,J$ \\
    \hline 
\end{tabular}
\end{table}
 

The amount of photons per second can then be calculated using \cref{eq:PPS}. The calculation is shown in

\begin{align}\label{eq:PPS}
\text{photon}/s = \frac{P}{E_{photon}}
\end{align}

\begin{table}[H]
\centering
\caption{Pulse frequency for both modes of operation}
\label{tab:photons_hitting_SPADs}
\begin{tabular}{|l|r|}\hline
    \textbf{photons hitting SPADs} & \\
    \hline 
    $P'_B$ & $7.76\cdot10^{-6}\,W$ \\
    $E_{photon}$ & $2.34\cdot10^{-19}\,J$ \\
    photons at SPADs & $3.31\cdot10^{13}\,\text{photon}/s$ \\
    \hline 
\end{tabular}
\end{table}












