\subsection{Detected photon characterisation} 
\label{ssec:detected_count_characterisation}
This section tackles the problem of calculating the resolution of the set-up. To do this, the different types of detection by the SPADs, due to sunlight, signal source and dark counts, must be characterized first.

To avoid pileup a couple of temporary assumptions about the performance of the SPADs will be made. Firstly, it is temporarily assumed that the percentage of the surface that is sensitive to photons is $50\,\%$. Secondly it is temporarily assumed that the dark count rate (DCR), of the entire array is $PSS_N=1\cdot10^9 \text{photon}/s$.

The amount of detected sunlight photons detected by the SPADs per second can now be calculated using \cref{eq:PPS_B}. The calculation are shown in \cref{tab:PPS}. Both the sunlight detections and the DCR detections are uniformly distributed.

\begin{align}\label{eq:PPS_B}
PPS_B = \text{photons at SPADs}\cdot \text{PDP} \cdot \text{effective area}
\end{align}

\begin{table}[H]
\centering
\caption{Pulse frequency for both modes of operation}
\label{tab:PPS}
\begin{tabular}{|l|r|}\hline
    \textbf{PPS for background photons} & \\
    \hline 
    $P_B2$ & $4.82\,p W$ \\
    $E_{photon}$ & $2.34\cdot10^{-19}\,J$ \\
    $PDP$ & $35.00\, \%$ \\
    $PPS_B$ & $7.21\cdot10^{6}\,$ \\
    \hline 
\end{tabular}
\end{table}


The next step is to characterize the relationship between the laser power and the received photons. It is assumed that all light emitted by the laser is hitting Europa. It is also assumed that the efficiency of the laser is $10\,\%$. Next, similar calculation as performed with $P_B$ can be used to calculate the amount of detected signal photons per second $PPS_S$. There is one important difference however: where for sunlight an altitude of $500\,m$ was used, for the signal photons an altitude of $8\,km$ must be considered. The calculation are performed in \cref{tab:PPS_S}. Note that even with the largest power budget of $50\,W$, the SNR will be well below $0\,dB$.

\begin{table}[H]
\centering
\caption{Amount of detected signal photons detected per second}
\label{tab:PPS_S}
\begin{tabular}{|l|r|}\hline
    \textbf{PPS for background photons} & \\
    \hline 
    $P_S$ & $1.00\,W$ \\
    Altitude & $8.00\, km$ \\
    $P'_S$ & $1.61\cdot10^{-11}\,W$ \\
    $PPS_S$ & $1.20\cdot10^{7}\,\text{counts}/s$ \\
    \hline 
\end{tabular}
\end{table}


The signal photons are not uniformly distributed like the noise and background photons. The signal photons are assumed to be distributed in a normal distribution with a $FWHM=100\,ps$. This distribution is convoluted with the jitter on the SPADs, which is assumed to be a normal distribution with a  $FWHM=100\,ps$. Using \cref{eq:sigma_FWHM} and \cref{eq:conv_sigma}, the resulting distribution has a $\sigma_S=60\,ps$. The mean $\mu_s$ is equal to the time of flight of the photon.

\begin{align}\label{eq:sigma_FWHM}
	FWHM   &= 2\sqrt{2\ln2}\sigma \\
	\sigma &= \frac{FWHM}{2\sqrt{2\ln2}}\label{eq:FWHM_sigma}
\end{align}

\begin{align}\label{eq:conv_sigma}
\sigma_{f\otimes g} = \sqrt{\sigma_f^2+\sigma_g^2}
\end{align}

The maximum allowable $FWHM$ of the measurement is $333\,ps$ as shown in \cref{ssec:assumptions}. Using \cref{eq:FWHM_sigma} the maximum standard deviation is $\sigma=141\,ps$.
















