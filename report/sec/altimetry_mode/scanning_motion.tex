\subsection{SPAD Array and Scanning Motion}
The specifications state the dimensions and resolution of the image. The specifications are provided in \cref{tab:image_properties}.

\begin{table}[H]
\centering
\caption{Properties of ground image}
\label{tab:image_properties}
\begin{tabular}{|l|ll|}\hline
\textbf{image properties}       &         &                     \\ \hline
ground area coverage   & 15625   & $m^2$ \\
ground sample distance & 0.05    & $m$                   \\
total amount of pixels & 6250000 & $-$ \\ \hline                   
\end{tabular}
\end{table}

\subsubsection*{Basic approach}
The most straightforward pixel layout, is one where every pixel is represented by a single SPAD. The size of a SPAD with a quenching circuit is assumed to be $20\times2\,\mu m^2$. This results in a chip size of $5\times5\, cm^2$, which is very large for a chip. This also only accounts for the SPAD and the quenching electronics. The TDCs and other components need to be put elsewhere. The system is also inefficient, because only a limited number of transmitted photons arrives at the SPADs, the SPADs are mostly idle, or receiving background noise. 

\subsubsection*{Scanning}
To tackle this issue one can use scanning. Here a SPAD is responsible for multiple pixels. This has a couple of advantages. First of all, the idle time of the SPADs goes down, which means that the SNR goes up. Second, the chip size can be a lot smaller. A clear negative is that a scanning motion must be made. Two types of scanning will be considered. The first one is a line of SPADs, and the second one a square of SPADs.

\subsubsection*{Horizontal Line layout}
In the horizontal line layout each SPAD is responsible for every point on a vertical line. The width of the chip remains the same at $5\,cm$, but the length is extremely short which besides production costs, leaves room for on board TDCs, that are directly attached to the SPADs. The Laser must be configured to send out a line, and the system must be capable of performing a vertical scanning motion. Also the lens must be configured to concentrate the incoming information on a line instead of a more square shaped form. The main problem of the scanning motion is that it should vary based on the altitude. This could cause uncertainty problems, especially at large speeds.

\subsubsection*{Square Layout}
The second alternative uses the same approach as the basic approach, but now with less SPADs. The ground image is divided into blocks that are scanned consecutively. The scanning motion is more complicated than both the basic, and horizontal line layout, but the main advantage is that the amount of SPADs are free to choose for the designer.

\subsubsection*{Comparison}
The first design with a SPAD for every pixel is not feasible. The size of the resulting chip would be to large, even in the most optimistic case, and the advantages for this design are mostly limited to the absence of the scanning motion. The Square layout is a better alternative in terms of chip design and efficiency, but the scanning motion is very complex, using two axis, and cause a lot of extra jitter on the signal. The Horizontal Line Layout is the most promising layout of the three designs. It uses only one axis, which is the most reasonable of the two scanning motions, and also has the advantage of a smaller and more efficient chip design. For the following calculations, the assumption will be made that the SPAD Array is of size $2500 \times 4$ for a total of $N_{SPAD} = 10000$.
