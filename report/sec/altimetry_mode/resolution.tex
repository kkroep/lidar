\subsubsection{Resolution} 
\label{sssec:resolution}

The maximum allowable $FWHM$ of the measurement is $333\,ps$ as shown in \cref{ssec:assumptions}. Using \cref{eq:FWHM_sigma} the maximum standard deviation is $\sigma=141\,ps$.

\begin{align}\label{eq:sigma_FWHM}
	FWHM   &= 2\sqrt{2\ln2}\sigma \\
	\sigma &= \frac{FWHM}{2\sqrt{2\ln2}}\label{eq:FWHM_sigma}
\end{align}

The power of the background noise leaving the surface of interest at europa is calculated in \cref{ssec:background_noise}, and is $P_B = 0.179\,W$. However, this is the power at Europa. One can use \cref{eq:europa_SPAD} to calculate the amount of power that hits the SPADs.

\begin{align}
	P_B' &= \frac{P_B}{2\pi r^2}\cdot R_{europa}\cdot \pi D_l\cdot L_fL_l\\
		 &= \frac{P_BR_{europa}D_lL_fL_l}{2r^2} \label{eq:europa_SPAD}
\end{align}
~\\
where $P_B'$ is the background noise at the SPADs, \\
$r$ the altitude of the sensor, \\
$R_{europa}$ the reflectivity of Europa, \\
$D_l$ the diameter of the lens, \\
$L_f$ the opacity of the bandpass filter, and \\
$L_l$ the effective opacity of the optics. This calculation is performed in \cref{tab:effective_noise_power}.

\begin{table}[H]
\centering
\caption{Pulse frequency for both modes of operation}
\label{tab:effective_noise_power}
\begin{tabular}{|l|r|}\hline
    \textbf{effective noise power} & \\
    \hline 
    $P_B$ & $1.89\,m W$ \\
    $r$ & $500.00\, m$ \\
    $R_{europa}$ & $35.00\, \%$ \\
    Diameter lens $(D_l)$ & $50.00\,m m$ \\
    opacity filter $(L_f)$ & $50.00\, \%$ \\
    opacity optics $(L_l)$ & $14.60\, \%$ \\
    $P_B2$ & $4.82\,p W$ \\
    \hline 
\end{tabular}
\end{table}


The next step is to calculate the number of background noise photons per second ($PPS_B$) that are detected by each SPAD. For that, one first needs to know the energy of a single photon $E_{photon}$. To calculate $E_{photon}$ one can use \cref{eq:energy_photon}. The calculation of this is shown in \cref{tab:energy_photon}.

\begin{align}\label{eq:energy_photon}
	E_{photon} &= \frac{hc}{\lambda}
\end{align}

\begin{table}[h]
\centering
\caption{Calculation of photon energy}
\label{tab:energy_photon}
\begin{tabular}{|l|ll|} \hline
\textbf{Energy of photon} &                     &       \\ \hline
$h$                       & $6.63\cdot10^{-34}$ & $Js$  \\
$c$                       & $3.00\cdot10^8$     & $m/s$ \\
$\lambda$                 & $850$               & $nm$  \\
$E_{photon}$              & $2.34\cdot10^{-19}$ & $J$   \\ \hline
\end{tabular}
\end{table}

The SPAD Array is assumed to be $2500\times4$ for a total of $N_{SPAD} = 10000$.
Now to calculate the $PPS_B$ for each SPAD one can use \cref{eq:PPS_B}

\begin{align}\label{eq:PPS_B}
	PPS_B &= \frac{P_B'\cdot PDP}{E_{photon}\cdot N_{SPAD}}
\end{align}
where $PDP$ is the photon detection probability of the SPADs. This calculation is done in \cref{tab:PPS_B}

\begin{table}[h]
\centering
\caption{Calculations of background photons per second}
\label{tab:PPS_B}
\begin{tabular}{|l|lll|} \hline
\textbf{PPS for background photons} & \textbf{Altimetry} & \textbf{Hazard Detection} &   \\ \hline
$P_B'$           & $5.69\cdot10^{-12}$ & $1.46\cdot^{-9}$     & $W$ \\
$E_{photon}$     & $2.34\cdot10^{-19}$ & $2.34\cdot10^{-19}$  & $J$ \\
$N_{SPAD}$       & $10000$             & $10000$              & $-$    \\
$PDP$            & $0.35$              & $0.35$               & $-$    \\
$PPS_B$          & $267$               & $6.84\cdot10^4$      & $-$    \\ \hline
\end{tabular}
\end{table}
