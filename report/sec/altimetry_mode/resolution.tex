\subsubsection{Resolution} 
\label{sssec:resolution}
This section tackles the problem of calculating the resolution of the set-up. To do this, the different types of detection by the SPADs, due to sunlight, signal source and dark counts, must be characterized first.

To avoid pileup a temporary assumption will be made about the performance of the SPADs. It is temporarily assumed that the percentage of the surface that is sensitive to photons is $50\,\%$. The amount of detected sunlight photons detected by the SPADs per second can now be calculated using

\begin{align}\label{eq:PPS_B}
PPS_B = \text{photons at SPADs}\cdot \text{PDP} \cdot \text{effective area}
\end{align}






The maximum allowable $FWHM$ of the measurement is $333\,ps$ as shown in \cref{ssec:assumptions}. Using \cref{eq:FWHM_sigma} the maximum standard deviation is $\sigma=141\,ps$.

\begin{align}\label{eq:sigma_FWHM}
	FWHM   &= 2\sqrt{2\ln2}\sigma \\
	\sigma &= \frac{FWHM}{2\sqrt{2\ln2}}\label{eq:FWHM_sigma}
\end{align}
